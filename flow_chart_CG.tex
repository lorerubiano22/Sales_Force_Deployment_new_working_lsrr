%\documentclass[x11names]{article}
%\usepackage{tikz}
%\usetikzlibrary{shapes,arrows,chains}
%%%<
%\usepackage{verbatim}
%\usepackage[active,tightpage]{preview}
%\PreviewEnvironment{tikzpicture}
%\setlength\PreviewBorder{5mm}%

%\begin{document}
% =================================================
% Set up a few colours
%\colorlet{lcfree}{Green3}
\colorlet{lcnorm}{OVGUblue4}
%\colorlet{lccong}{Red3}
% -------------------------------------------------
% Set up a new layer for the debugging marks, and make sure it is on
% top
\pgfdeclarelayer{marx}
\pgfsetlayers{main,marx}
% A macro for marking coordinates (specific to the coordinate naming
% scheme used here). Swap the following 2 definitions to deactivate
% marks.
\providecommand{\cmark}[2][]{%
  \begin{pgfonlayer}{marx}
    \node [nmark] at (c#2#1) {#2};
  \end{pgfonlayer}{marx}
  } 
\providecommand{\cmark}[2][]{\relax} 
% -------------------------------------------------
% Start the picture
\begin{tikzpicture}[%
    >=triangle 60,              % Nice arrows; your taste may be different
    start chain=going below,    % General flow is top-to-bottom
    node distance=6mm and 60mm, % Global setup of box spacing
    every join/.style={norm},   % Default linetype for connecting boxes
    ]
% ------------------------------------------------- 
% A few box styles 
% <on chain> *and* <on grid> reduce the need for manual relative
% positioning of nodes
\tikzset{
  base/.style={draw, on chain, on grid, align=center, minimum height=8ex},
  proc/.style={base, rectangle, text width=16em},
  test/.style={base, diamond, aspect=2, text width=5em},
  term/.style={proc, rounded corners},
  % coord node style is used for placing corners of connecting lines
  coord/.style={coordinate, on chain, on grid, node distance=6mm and 50mm},
  % nmark node style is used for coordinate debugging marks
  nmark/.style={draw, cyan, circle, font={\sffamily\bfseries}},
  % -------------------------------------------------
  % Connector line styles for different parts of the diagram
  norm/.style={->, draw, lcnorm},
  free/.style={->, draw, lcfree},
  cong/.style={->, draw, lccong},
  it/.style={font={\small\itshape}}
}

\node [term]      {\textbf{Start: }$\forall i \in I, j \in J_i$, initialize $T_{ij} = \mathcal{U}(0,0.01) \cdot T$};
\node [term,join](t2) {Solve Restricted Master Problem};
%\node [term,join] {$\forall i \in I, j \in J_i$, compute reduced costs $\overline{p}_{ij}(t)$ of $x_{ijt}$};
\node [term,join] {\textbf{Subproblem:} $\forall i \in I, j \in J_i$, compute optimal selling time $t^*_{ij}$ that maximizes reduced cost $\overline{p}_{ij}(t)$ of variable $x_{ijt}$  };
%\node [term,join](t5) {$\forall i \in I, j \in J_i$, };
\node [test,join] (t6){$\forall i \in I, j \in J_i: \overline{p}_{ij}(t^*) \leq 0$};
%\node [right=of t1] (t2) {};
\node [term, right= 75mm of t6] (y1) {$\forall i \in I, j \in J_i$: if $\overline{p}_{ij}(t^*)>0$ then $T_{ij}:=T_{ij} \cup \{t^*\}$ };
\node [coord, right= 75mm of t2] (c1)  {};
\draw [->,lcnorm] (t6) to node[above,black]{no} (y1);
\draw [-,lcnorm] (y1) -- (c1) ;
\draw [->,lcnorm] (c1) -- (t2) ;
\node [term, below=25mm of t6](t7) {\textbf{Stop: }optimal solution for the Master Problem (LP-Relaxation)};
\draw [->,lcnorm,pos=0.35](t6) to node[left,black]{yes} (t7);
\end{tikzpicture}

%\end{document}